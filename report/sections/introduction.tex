\section{Introduction}
This project aims to provide an interactive application, InFlight, that allows users to explore and analyze the \textit{Airline Reporting Carrier On-Time Performance} dataset \cite{usdot2023}. The dataset contains information on the on-time performance of domestic flights operated by large air carriers in the United States between 1987 and 2020. The application is divided into three parts, each with a set of features that allow users to explore different aspects of the dataset. The following sections will detail the objectives, accomplishments, and features implemented in each part of the project.

\subsection{Project Objectives}
The main objectives of the project is to achieve the following:
\begin{enumerate}
    \item Develop a user-friendly interface that provides visualizations and insights into the dataset.
    \item Gain familiarity with PySpark and the PySpark SQL API for processing large datasets.
    \item Gain valuable insights into the on-time performance of domestic flights in the United States.
\end{enumerate}

\subsection{Accomplishments}
The following accomplishments have been achieved in the project:
\begin{enumerate}
    \item Implemented a web-based application to interact with the dataset. This application utilizes Flask and PySpark as the backend and ReactJS and Tailwind CSS as the frontend.
    \item Gained sufficient knowledge of PySpark and the PySpark SQL API by successfully completing all three parts of the project.
    \item Developed a set of features that allow users to explore and analyze different aspects of the dataset, such as flight performance metrics, regional performance, and airline performance.
\end{enumerate}

\subsubsection{Implemented Features}
The project is divided into three parts, each with a set of features that are implemented. The features implemented in each part are shown in Table 1 below.

\begin{table}[H]
    \centering
\begin{tabular}{|l|l|l|}
\hline
\textbf{Part 1}                                                                                                                           & \textbf{Part 2}                                                                                                                                                                        & \textbf{Part 3} \\ \hline
\begin{tabular}[c]{@{}l@{}}Read in and store the dataset\\ using PySpark\end{tabular}                                            & \begin{tabular}[c]{@{}l@{}}Determine a "performance" metric\\ for airports and display a radar chart\\ to compare airport performance given\\ a list of airports\end{tabular} &        \\ \hline
\begin{tabular}[c]{@{}l@{}}Display the total number of \\ flights given a year\end{tabular}                                      & \begin{tabular}[c]{@{}l@{}}Chart region and state performance on\\ on a US map\end{tabular}                                                                                   &        \\ \hline
\begin{tabular}[c]{@{}l@{}}Display the total number of \\ flights given a range of years\end{tabular}                            & \begin{tabular}[c]{@{}l@{}}Display a table of the best and worst\\ performing states in each US region\end{tabular}                                                           &        \\ \hline
\begin{tabular}[c]{@{}l@{}}Display the total number of\\ flights given comma-separated\\ list of years\end{tabular}              &                                                                                                                                                                               &        \\ \hline
\begin{tabular}[c]{@{}l@{}}Display the percentage of flights\\ that departed on time, early and\\ late\end{tabular}              &                                                                                                                                                                               &        \\ \hline
\begin{tabular}[c]{@{}l@{}}Display the top reason for cancelled\\ flights given a year\end{tabular}                              &                                                                                                                                                                               &        \\ \hline
\begin{tabular}[c]{@{}l@{}}Display the top 3 airports with the \\most punctual flights for a given year\end{tabular}                              &                                                                                                                                                                               &        \\ \hline
\begin{tabular}[c]{@{}l@{}}Determine a "performance" metric\\ and display the top three worst\\ performing airlines\end{tabular} &                                                                                                                                                                               &        \\ \hline
\end{tabular}
\label{tab:tasks}
\caption{Features implemented in each part of the project.}
\end{table}

\subsection{Organization of the Report}
The remainder of this report is organized in the following manner:

\begin{itemize}
    \item \textbf{Part 1: Core Features} - Details the core features implemented in the project, including reading in and storing the dataset, displaying flight statistics, and analyzing flight timeliness.
    \item \textbf{Part 2: Intermediate Features} - Details the intermediate features implemented in the project, including airport performance metrics, regional performance analysis, and state performance analysis.
    % TODO : FIX PART 3
    \item \textbf{Part 3: Advanced Features} - 
    \item \textbf{Evaluation} - Evaluates the project based on the objectives and accomplishments of the project, along with a reflection on lessons learned and experience gained from the project.
    \item \textbf{Conclusion} - Concludes the report with a summary of the project and future work that can be done to improve the application.

